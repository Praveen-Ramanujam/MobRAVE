\hypertarget{orcollision.cpp-example}{
\subsection{orcollision.cpp}
}
\begin{DoxyAuthor}{Author}
Rosen Diankov
\end{DoxyAuthor}
Load a robot into the openrave environment, set it at \mbox{[}joint values\mbox{]} and check for self collisions. Returns number of contact points.

Usage: \begin{DoxyVerb}
    orcollision [--list] [--checker checker_name] [--joints #values [values]] body_model
    \end{DoxyVerb}



\begin{DoxyItemize}
\item {\bfseries -\/-\/list} -\/ List all the loadable interfaces (ie, collision checkers).
\item {\bfseries -\/-\/checker} -\/ name Load a different collision checker instead of the default one.
\item {\bfseries -\/-\/joints \#values \mbox{[}values\mbox{]}} -\/ Set the robot to specific joint values
\end{DoxyItemize}

Example: \begin{DoxyVerb}
    orcollision --checker ode robots/barrettwam.robot.xml
    \end{DoxyVerb}


{\bfseries Full Example Code:}


\begin{DoxyCodeInclude}

#include <openrave-core.h>
#include <vector>
#include <cstring>
#include <sstream>

using namespace OpenRAVE;
using namespace std;


void printhelp()
{
    RAVELOG_INFO("orcollision [--list] [--checker checker_name] [--joints #values
       [values]] body_model\n");
}

void printinterfaces(EnvironmentBasePtr penv)
{
    std::map<InterfaceType, std::vector<std::string> > interfacenames;
    RaveGetLoadedInterfaces(interfacenames);
    stringstream ss;

    ss << endl << "Loadable interfaces: " << endl;
    for(std::map<InterfaceType, std::vector<std::string> >::iterator itinterface 
      = interfacenames.begin(); itinterface != interfacenames.end(); ++itinterface) {
        ss << RaveGetInterfaceName(itinterface->first) << "(" << itinterface->sec
      ond.size() << "):" << endl;
        for(vector<string>::iterator it = itinterface->second.begin(); it != itin
      terface->second.end(); ++it)
            ss << " " << *it << endl;
        ss << endl;
    }
    RAVELOG_INFO(ss.str());
}

int main(int argc, char ** argv)
{
    if( argc < 2 ) {
        printhelp();
        return -1; // no robots to load
    }

    RaveInitialize(true); // start openrave core
    EnvironmentBasePtr penv = RaveCreateEnvironment(); // create the main environ
      ment
    vector<dReal> vsetvalues;

    // parse the command line options
    int i = 1;
    while(i < argc) {
        if((strcmp(argv[i], "-h") == 0)||(strcmp(argv[i], "-?") == 0)||(strcmp(ar
      gv[i], "/?") == 0)||(strcmp(argv[i], "--help") == 0)||(strcmp(argv[i], "-help") =
      = 0)) {
            printhelp();
            return 0;
        }
        else if( strcmp(argv[i], "--checker") == 0 ) {
            // create requested collision checker
            CollisionCheckerBasePtr pchecker = RaveCreateCollisionChecker(penv,ar
      gv[i+1]);
            if( !pchecker ) {
                RAVELOG_ERROR("failed to create checker %s\n", argv[i+1]);
                return -3;
            }
            penv->SetCollisionChecker(pchecker);
            i += 2;
        }
        else if( strcmp(argv[i], "--list") == 0 ) {
            printinterfaces(penv);
            return 0;
        }
        else if( strcmp(argv[i], "--joints") == 0 ) {
            vsetvalues.resize(atoi(argv[i+1]));
            for(int j = 0; j < (int)vsetvalues.size(); ++j)
                vsetvalues[j] = atoi(argv[i+j+2]);

            i += 2+vsetvalues.size();
        }
        else
            break;
    }

    if( i >= argc ) {
        RAVELOG_ERROR("not enough parameters\n");
        printhelp();
        return 1;
    }

    // load the scene
    if( !penv->Load(argv[i]) ) {
        return 2;
    }

    // lock the environment to prevent thigns from changes
    EnvironmentMutex::scoped_lock lock(penv->GetMutex());

    vector<KinBodyPtr> vbodies;
    penv->GetBodies(vbodies);
    // get the first body
    if( vbodies.size() == 0 ) {
        RAVELOG_ERROR("no bodies loaded\n");
        return -3;
    }

    KinBodyPtr pbody = vbodies.at(0);
    vector<dReal> values;
    pbody->GetDOFValues(values);

    // set new values
    for(int i = 0; i < (int)vsetvalues.size() && i < (int)values.size(); ++i) {
        values[i] = vsetvalues[i];
    }
    pbody->SetDOFValues(values,true);

    int contactpoints = 0;
    CollisionReportPtr report(new CollisionReport());
    penv->GetCollisionChecker()->SetCollisionOptions(CO_Contacts);
    if( pbody->CheckSelfCollision(report) ) {
        contactpoints = (int)report->contacts.size();
        stringstream ss;
        ss << "body in self-collision "
           << (!!report->plink1 ? report->plink1->GetName() : "") << ":"
           << (!!report->plink2 ? report->plink2->GetName() : "") << " at "
           << contactpoints << "contacts" << endl;
        for(int i = 0; i < contactpoints; ++i) {
            CollisionReport::CONTACT& c = report->contacts[i];
            ss << "contact" << i << ": pos=("
               << c.pos.x << ", " << c.pos.y << ", " << c.pos.z << "), norm=("
               << c.norm.x << ", " << c.norm.y << ", " << c.norm.z << ")" << endl
      ;
        }

        RAVELOG_INFOA(ss.str());
    }
    else {
        RAVELOG_INFO("body not in collision\n");
    }

    // get the transformations of all the links
    vector<Transform> vlinktransforms;
    pbody->GetLinkTransformations(vlinktransforms);

    penv->Destroy(); // destroy
    return contactpoints;
}
\end{DoxyCodeInclude}
 